\section{Introduction}
\begin{frame}{Introduction}
\tikzstyle{block} = [draw, fill=blue!20, rectangle, 
    minimum height=3em, minimum width=6em]
\tikzstyle{sum} = [draw, fill=blue!20, circle, node distance=1cm]
\tikzstyle{input} = [coordinate]
\tikzstyle{output} = [coordinate]
\tikzstyle{pinstyle} = [pin edge={to-,thin,black}]

% The block diagram code is probably more verbose than necessary
\begin{center}
\begin{tikzpicture}[auto, node distance=2cm,>=latex']
    % We start by placing the blocks
    \node [input, name=input] {};
    \node [block, right of=input, pin={[pinstyle]above:Disturbances},
            node distance=2cm] (system) {System};
    \node [output, right of=system] (output) {};

    % Once the nodes are placed, connecting them is easy. 
    \draw [draw,->] (input) -- node {$x$} (system);
    \draw [->] (system) -- node [name=y] {$y^e(x)$} (output) ;
\end{tikzpicture}
\end{center}
\pause

\begin{center}
% The block diagram code is probably more verbose than necessary
\begin{tikzpicture}[auto, node distance=2cm,>=latex']
    % We start by placing the blocks
    \node [input, name=input] {};
    \node [block, right of=input, pin={[pinstyle]above:Disturbances},
            node distance=2cm] (system) {System};
     \node [block, below of=system] (model) {Model};
    \node [sum, right of=output] (sum) {};
    \node [output, right of=sum] (final_output) {};

    % Once the nodes are placed, connecting them is easy. 
    \draw [draw,->] (input) -- node [name=x] {$x$} (system);
    \draw [->] (system) -- node [name=y] {$y^e(x)$} (sum) ;
    \draw [->] (x) |- (model);
    \draw [->] (model) -| node [near end] {$y^m(x)$} (sum);
    \draw [->] (sum) -- node [name=y_final] {$\delta(x)$} (final_output) ;
\end{tikzpicture}
\end{center}
\end{frame}

\begin{frame}{Motivation}{Inverse Uncertainty Quantification}
	\begin{block}{Bias Correction}
		\begin{align}
			y^e(x) = y^m(x)+\delta(x)+\epsilon \nonumber
		\end{align}
		where $\epsilon$ is the experimental uncertainty.
	\end{block}
	% Model Inadequacy
	\pause
	\begin{block}{Parameter Calibration}
		\begin{align}
			y^e(x) = y^m(x,\theta^*)+\epsilon \nonumber
		\end{align}
	\end{block}
	\pause
	\begin{block}{Bias Correction and Parameter Calibration}
		\begin{align}
			y^e(x) = y^m(x,\theta^*)+\delta(x)+\epsilon \nonumber
		\end{align}
	\end{block}

\end{frame}

\begin{frame}{Need for Function Approximation}
	\begin{itemize}
		\item Lack of simulation results $y^m(x,\theta)$
		\item Parameterizing discrepancy function $\delta(x)$
	\end{itemize}

	\begin{block}{Why Bayesian}
		\begin{itemize}
			\item Integration of prior knowledge

		\end{itemize}
	\end{block}
	\vskip0pt plus 1filll
Bayesian calibration of computer models, Marc C. Kennedy, Anthony O'Hagan in Journal of the Royal Statistical Society, 2001 

\end{frame}
